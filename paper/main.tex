\documentclass[11pt,a4paper]{article}
\usepackage[margin=1in]{geometry}
\usepackage{amsmath,amssymb,amsthm}
\usepackage{bm}
\usepackage{physics}
\usepackage{hyperref}
\usepackage{graphicx}
\usepackage{microtype}
\usepackage{mathtools}
\usepackage{siunitx}

\title{Arc Vector Algebra (AVA): An Intuitive Path into Differential Geometry}
\author{Sanjin}
\date{\today}

\begin{document}
\maketitle

\begin{abstract}
We introduce Arc Vector Algebra (AVA), a calculus built around arc vectors on (possibly evolving) manifolds.
AVA aims to provide a practical language for trajectory planning and physical modeling, with connections to
classical differential geometry and control. This document is a living whitepaper.
\end{abstract}

\section{Motivation}
Brief summary of why AVA is helpful compared to purely point-based or tangent-vector-only formalisms.

\section{Core Objects}
Define arc vectors, the arc gradient $\nabla_a$, and any associated operators.
State assumptions and minimal axioms.

\section{Geometry and Dynamics}
Relate AVA objects to geodesics, curvature, and flows on a manifold.
If desired, outline how AVA interfaces with a displacement interpretation of spacetime.

\section{Worked Example}
Sketch an example (e.g., Earth--Mars transfer) showing how AVA is used in practice.

\section{Outlook}
Open questions, planned proofs, and computational approaches.

\bibliographystyle{abbrv}
\bibliography{refs}
\end{document}
